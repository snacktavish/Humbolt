\documentclass[10pt]{article}
\usepackage[pass,letterpaper]{geometry}
% Text layout

% Text layout
\topmargin 0.0cm
\oddsidemargin 0.5cm
\evensidemargin 0.5cm
\textwidth 16cm 
\textheight 21cm

\usepackage{fixltx2e}
%opening
\title{Key Publications}
\author{Emily Jane McTavish}
%\date{\vspace{-5ex}}
\date{}
\begin{document}

\maketitle
%[Please explain your selection of key publications here (relevance of results, significance for your
%academic profile; maximum of 1000 characters). In cases of multiple authorship please also indicate
%your own personal contribution to this key publication.]

\textbf{McTavish, E.J.}, Hillis, D.M. How does ascertainment bias in SNP analyses affect inferences about population history? Submitted to Molecular Ecology (publisher’s acknowledgement of receipt attached)\\

Single nucleotide polymorphism (SNP) data sets are valuable and readily obtained using chip based genotyping by sequencing.
However, biases in the loci chosen for these studies can impact population genetic inferences. 
We used extensive simulations under a variety of demographic scenarios, and selected SNPs from the simulated data sets under biased and non biased conditions. 
We compared the inference of population histories form those data to empirical data. 
F\textsubscript{ST} was downwardly biased by inflated within group variation under sub-population biased ascertainment schemes. 
While ascertainment bias can affect inference of admixture among populations using principal components analysis (PCA), this method was largely robust to even strong bias.\\

\textbf{McTavish, E.J.}, Hillis, D.M. in press. A genomic approach for distinguishing between recent and ancient admixture in cattle.  \textsl{Journal of Heredity}.(publisher’s letter of acceptance attached)\\

As dense marker sampling across the genome is becoming more common, methods are being developed to leverage information about linkage among markers to understand population structure. 
We extended a technique for mapping regions of ancestry across the genome, previously only applied to humans and and chimpanzees, to understanding admixture in 4 hybrid cattle breeds. 
We developed a simple metric, scaled block size (SBS) which uses the relative sizes of non-recombined blocks of introgressed genetic material to compare timing of admixture among groups. 
Using this metric we were able to distinguish between individuals with recent and ancient admixture, even when the ancestry proportions were identical.\\

\textbf{McTavish, E.J.}, Decker, J.E., Schnabel, R.D., Taylor, J.F., Hillis, D.M. 2013. New World cattle show ancestry from multiple independent domestication events.  \textsl{Proceedings of the National Academy of Sciences USA}. 110:E1398-E1406.\\

In this publication we applied genomic SNP data generated by Decker, Schnabel, and Taylor to understanding global cattle population structure, with a focus on the ancestry of new world cattle. 
We found that although the two domesticated subspecies of cattle, taurine and indicine, are deeply diverged (most recent common ancestor greater than 200,000 years ago), there has been extensive gene flow between these lineages. 
While a cline of admixture was known to have been present across northern Africa, we found that indicine introgression is also present in most New World derived cattle breeds. This suggests that admixture was present in the first cattle brought to North America by Spanish colonists in 1493.
\end{document}
