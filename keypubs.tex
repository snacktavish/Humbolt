\documentclass[a4paper,10pt]{article}
\usepackage[utf8]{inputenc}

%opening
\title{Key Publications}
\author{Emily Jane McTavish}

\begin{document}

\maketitle
%[Please explain your selection of key publications here (relevance of results, significance for your
%academic profile; maximum of 1000 characters). In cases of multiple authorship please also indicate
%your own personal contribution to this key publication.]

\textbf{McTavish, E.J.}, Hillis, D.M. submitted. How does ascertainment bias in SNP analyses affect inferences about population history? (publisher’s acknowledgement of receipt attached)

Although single nucleotide polymorphism (SNP) data sets are valuable and readlily optained using chip based geotyping by sequencing, biases in the loci chosen for these studies can impact population genetic inferences.
We used extensive simulations under a variety of demographic scenarios, and selected SNPs from the simulated data sets under biased and non biased conditiaons.
WE compared the inference of poulation histories form those data to empirical data. W

\textbf{McTavish, E.J.}, Hillis, D.M. in press. A genomic approach for distinguishing between recent and ancient admixture in cattle.  \textsl{Journal of Heredity}.(publisher’s letter of acceptance attached)

As dense marker sampling across the genome is getting more common in humans, methods are being developed to leverage informtion about population structure and linkage among markers.
We extended a technique for mapping regions of ancestry across the genome, previously only applied to humans and and chimpanzees, to understanding admixture in 4 hybrid cattle breeds.


\textbf{McTavish, E.J.}, Decker, J.E., Schnabel, R.D., Taylor, J.F., Hillis, D.M. 2013. New World cattle show ancestry from multiple independent domestication events.  \textsl{Proceedings of the National Academy of Sciences USA}. 110:E1398-E1406.

In this publication we applied genomic SNP data generated by Decker, Schnabel, and Taylor to understanding global cattle population structure, with a fucus on the ancestry of new world cattle.


\end{document}
